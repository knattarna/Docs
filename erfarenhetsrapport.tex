\documentclass{article}
%settings
\usepackage[utf8]{inputenc}
\usepackage[top=2in, bottom=1.5in, left=1in, right=1in]{geometry} %page margins

%custom commands
\newcommand{\HRule}{\rule{\linewidth}{0.5mm}}

\begin{document}

%TITLEPAGE
%\author{Knattarna} % The authors name
%\title{Erfarenhetsrapport TDDI02} % The title of the document
%\date{\today{}} % Sets date you can remove \today{} and type a date manually
%\maketitle{} % Generates title
\begin{titlepage}
\center

\textsc{\LARGE LiU}\\[1.5cm]
\textsc{\Large TDDI02 Programmeringsprojekt}\\[0.5cm]
\textsc{\large Knattarna}\\[0.5cm]

\HRule\\[0.4cm]
{\huge Erfarenhetsrapport}\\[0.4cm]
\HRule\\[1.4cm]

%Authors supervisor
\begin{minipage}{0.4\textwidth}
\begin{flushleft} \large
\emph{Knattar:}\\
Jonathan \textsc{Möller}\\
Patrick \textsc{Klimek}\\
Anton \textsc{Sundkvist}\\
Jacob \textsc{Hedén}\\
\end{flushleft}
\end{minipage}
% supervisor
\begin{minipage}{0.4\textwidth}
\begin{flushright} \large
\emph{Handledare:} \\
Rebecka \textsc{Geijer Michaeli} % Supervisor's Name
\end{flushright}
\end{minipage}\\[4cm]





%date
{\large \today}\\[0.3cm] % Date, change the \today to a set date if you want to be precise
{\large November 14, 2014}\\[3cm] % Date, change the \today to a set date if you want to be precise



\end{titlepage}


%END TITLEPAGE

\newpage

\section*{-v46}

%Jonathan
\paragraph{Jonathan}
När man skriver en designspecifikation vill man ha ett väldigt ordnat, välstrukturerat dokument, en ritning.
Det kan därför vara bra att försöka få en översiktsbild av hur dokumentet ska se ut innan man börjar skriva på det.
Då vi kanske kastade oss in på att börja skriva känns resultatet något oordnat och inte helt klart.
\\
I vårat projekt finns det mycket som måste studeras innan en bra implementation kan påbörjas. Vi måste lära oss en IDE, ett
nytt programmeringspråk samt ett API. Vi visste innan att all undersökning och informationssökning skulle ta mycket tid och försena
kodningen, men såklart underskattade vi mängden vi var tvungna att lära oss, vilket leder till förseningar i planeringen.
I framtida projekt kommer jag lägga mycket krut i de tidiga stadierna av projektet för att ta reda på vilka verktyg som kommer
användas och hur mycket tid som måste spenderas för att lära sig dessa. ( !OBS jag byter tempus här för att vissa erfarenheter är i presens :O ska ändras sen )

%patrick
\paragraph{Patrick}
Efter Jonas föreläsningar har jag lärt mig hur en professionell Software engineering-process ser ut. Om man enbart ska välja ett steg ur Software Engineering tycker jag det är kravspecifikationen. Jag tror man kan klara ett helt mjukvaru-projekt med bara en kravspecifikation, även om det är rekommenderat att göra alla delar. Ett realistiskt programmeringsprojekt har sin grund i kundens behov. I vårt fall är vi själva kunden så vi fick ställa kraven på produkten som vi skulle utveckla. Man skulle tro att det inte behövs en kravspecifikation om vi själva vet vad vi vill göra. Men det är snarare tvärtom. Tack vare kravspecifikationen får man inramning av projektet. Egentligen är det inget nytt för oss, de flesta programmeringslabbar vi gjort har någon form av specificerade krav. Kravspecifikationen måste alltid vara verfierbar, d.v.s. man bör inte använda adjektiv som "produkten ska vara rolig" eller "produkten ska har en snygg design". Istället ska man använda tekniskt avgränsade krav, som t.ex. "Sökningen i databasen ska inte ta mer än 10 sekunder".

%Anton
\paragraph{Anton}
Jag har lärt mig vikten av bra planering när vi arbetade med krav- och designspec. Dock så var det väldigt svårt att skapa en bra designspec med tanke på att ingen av oss i gruppen tidigare arbetat med Java eller Android Studio. Jag kommer med tiden få mer erfarenheter i Java och Android-systemet, förhoppningsvis kan jag ta med mig dessa kunskaper för framtida Androidrelaterade projekt.
Vi har skapat ett projekt på Github och integrerat en repository i Android Studio, så nu har vi fått viss erfarenhet med Git.

%Jacob
\paragraph{Jacob}
Svårt att skriva en bra designspecifikation när man inte har gjort en app tidigare.
Redan nu när man har lärt sig en liten liten del hur en app görs kan man skriva den 10x bättre. 
Lärt sig använda många bra verktyg som git, android studios och redmine.
Sedan att lära sig java samt android studios är något som kommer på gå hela projektet vilket även är vårat syfte med projektet.


\newpage{}
\section*{v47}

\end{document}
